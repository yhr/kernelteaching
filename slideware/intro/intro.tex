\documentclass{beamer}

\mode<presentation> {
\usecolortheme{seagull}

\setbeamertemplate{footline}[page number]}
\usepackage{graphicx} 
\usepackage{booktabs} 
\usepackage{url}

%-------------------------------------------------------------------------

\title[TLKDCC Intro]{The Linux Kernel Development Crash Course}
\author{Hans Holmberg}
\institute[LKTP]
{
Linux Kernel Teaching Project \\ 
\medskip
\textit{hans.holmberg@gmail.com}
}
\date{\today}
%------------------------------------------------------------------------
\begin{document}

\begin{frame}
\titlepage
\includegraphics{../common/tux}
\end{frame}

\begin{frame}
\frametitle{Overview}
\tableofcontents 
\end{frame}

\section{Background}
\begin{frame}
\frametitle{Background}
Why teaching kernel development?
\begin{itemize}
\item Kernel development has a steep learning curve
\item A crash course get students past the first bump
\end{itemize}

Why do it open-source?
\begin{itemize}
\item Share the effort!
\item More teachers, reach more students
\end{itemize}
\end{frame}

%-----------------------------------------------------------------------
\section{Course goals} 
\begin{frame}
\frametitle{Course goals - what is achievable?}
Provide a wide overview of the Linux kernel machinery and how to develop kernel code \\
.. and in the end turn students into kernel hackers, capable of contributing upstream and independently continue exploring.
\end{frame}

%-----------------------------------------------------------------------
\section{Prerequisites}

\begin{frame}
\frametitle{Prerequisites}
\begin{itemize}
	\item Mandatory
	\begin{itemize}
		\item C
		\item Operating system concepts
	\end{itemize}

	\item Optional (will help those who need it)
	\begin{itemize}
		\item GIT
		\item Build toolchain
		\item Linux tools(grep, diff, ..)
	\end{itemize}
\end{itemize}
\end{frame}

%-----------------------------------------------------------------------
\section{Lectures}
\begin{frame}
\frametitle{Lectures}
Theory 
\begin{itemize}
	\item Linux Kernel overview 
	\item Kernel boot and machine descriptions (ACPI/DT)
	\item Interrupts, threads and processes
	\item Power management
\end{itemize}
Craft
\begin{itemize}
	\item Navigating the source tree
	\item Configuring and building the kernel
	\item Debugging and tracing
\end{itemize}
Community
\begin{itemize}
	\item Development best practices
	\item Community interaction and upstreaming
\end{itemize}
\end{frame}

%-----------------------------------------------------------------------
\section{Labs}
\begin{frame}
\frametitle{Labs}
\begin{itemize}
	\item Configuring and building your own kernel
	\item Finding documentation in the source tree and history
	\item Debugging
	\item Driver development
	\item Learning a framework
	\item Upstreaming
\end{itemize}
\end{frame}

%-----------------------------------------------------------------------
\section{What's next?}
\begin{frame}
\frametitle{What's next?}
\begin{itemize}
	\item Great feedback so far, so we're going to continue this.
	\item Contributions very welcome
	\item Reviewers of the material and teachers wanted
	\item Want to attend the course? Let us know!
\end{itemize} 
\end{frame}

\begin{frame}
\frametitle{Contacts}
\begin{itemize}
	\item github.com/yhr/kernelteaching
	\item kernel-teaching@googlegroups.com
\end{itemize}
\end{frame}


%-----------------------------------------------------------------------
\begin{frame}
\Huge{\centerline{Thanks!}}
\end{frame}

\end{document} 
